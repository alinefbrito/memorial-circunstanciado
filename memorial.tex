%% Template adapted by Aline firmino brito
%% This document is based on latex template from Prof. DR Dr. Vinicius Garcia, adapted from Prof Daniel Cunha.
%%
%% Reference commands. Use the following commands to make references in your
%% text:
%%          \figref  -- for Figure reference
%%          \tabref  -- for Table reference
%%          \eqnref  -- for equation reference
%%          \chapref -- for chapter reference
%%          \secref  -- for section reference
%%          \appref  -- for appendix reference
%%          \axiref  -- for axiom reference
%%          \conjref -- for conjecture reference
%%          \defref  -- for definition reference
%%          \lemref  -- for lemma reference
%%          \theoref -- for theorem reference
%%          \corref  -- for corollary reference
%%          \propref -- for proprosition reference
%%          \pgref   -- for page reference
%%
%%          Example: See \chapref{chap:introduction}
%%%%%%%%%%%%%%%%%%%%%%%%%%%%%%%%%%%%%%%%%%%%%%%%%%%%%%%%%%%%%%%%%%%%%%%%%%%%%%%

\documentclass[a4paper,oneside,12pt]{article}
\usepackage{textcomp}
\usepackage{gensymb}
\usepackage{graphicx}
\usepackage{amsmath,amstext,amssymb,amsfonts}
\usepackage[none]{hyphenat}
\usepackage{fancyhdr}
\usepackage{cite}
\usepackage{indentfirst}
%\usepackage{path}
\usepackage[usenames, dvipsnames]{color}
%\usepackage[portuguese]{babel}
\usepackage[pt-BR]{datetime2}
\DTMlangsetup*{showdayofmonth=false}
\usepackage{pdfsync}
\usepackage{url}
\usepackage{setspace}
\usepackage[latin1,utf8]{inputenc}
\usepackage[T1]{fontenc}
\usepackage{colortbl}

\usepackage{booktabs}
\usepackage{ctable}
\usepackage{setspace}
\usepackage[multidot]{grffile}
\usepackage[final]{pdfpages}

\usepackage{titlesec}

\usepackage[toc,page]{appendix}

\newcommand{\SetRowColor}[1]{\noalign{\gdef\RowColorName{#1}}\rowcolor{\RowColorName}}


\definecolor{battleshipgrey}{rgb}{0.52, 0.52, 0.51}
	\definecolor{aurometalsaurus}{rgb}{0.43, 0.5, 0.5}
\newcommand{\appref}[1]{\@appendixname~\ref{#1}\xspace}

%\usepackage{xifthen}
\newcommand{\Title}{Memorial Circunstanciado}
\newcommand{\Author}{Nome Completo Autor}
\newcommand{\Subject}{Memorial Circunstanciado e documentação comprobatória solicitados como requisitos para participação no objetivo do documento. }
\newcommand{\SubItemB}[2]{
    {\setlength\itemindent{15pt} \item[-] \textbf{#1} {#2}}}
    
   
    \newcommand{\SubItemBB}[1]{
    {\setlength\itemindent{15pt} \item[-] \textbf{Escola: } #1}}

\newcommand{\Instituicao}[1]{
    {\setlength\itemindent{15pt} \item[-] {\textbf{Instituição$:$} #1}}}
    
\newcommand{\Conclusao}[1]{
    {\setlength\itemindent{15pt} \item[-] \textbf{Conclusão $:$} {#1}}}
    
  \newcommand{\RegDiploma}[1]{
    {\setlength\itemindent{15pt} \item[-] \textbf{N\degree de Registro do $:$} {#1}}}

   \newcommand{\Duracao}[1]{
    {\setlength\itemindent{15pt} \item[-] \textbf{N\degree  Horas:} {#1}}}   
    
   \newcommand{\DiscSup}[1]{
    {\setlength\itemindent{15pt} \item[-] \textbf{Disciplinas Ensino Superior:} {#1}}}      
    
     \newcommand{\DiscMedio}[1]{
    {\setlength\itemindent{15pt} \item[-] \textbf{Disciplinas Ensino Médio:} {#1}}} 
    
       \newcommand{\Periodo}[2]{
    {\setlength\itemindent{15pt} \item[-] \textbf{Período:} #1 à #2} }
    
    
    \newcommand{\desc}[2]{ \item[-] \textcolor{red}{\textbf{#1}}~\emph{#2}}
    
\usepackage[bookmarks,colorlinks,pdfpagelabels,
pdftitle={\Title}, pdfauthor={\Author},
pdfsubject={ \Subject }, pdfcreator={\Author}, pdfkeywords={    Docente}]{hyperref}
\hypersetup{
    colorlinks = true,
  linkcolor= battleshipgrey,
   citecolor=battleshipgrey.
   urlcolor  = aurometalsaurus
}

\newcommand{\otoprule}{\midrule[\heavyrulewidth]}

% Definição de margens
\setlength{\textwidth}{16cm} %
\setlength{\textheight}{23cm} %
\setlength{\oddsidemargin}{0cm} %
\setlength{\evensidemargin}{0cm} %
\setlength{\topmargin}{0cm}

\renewcommand{\abstractname}{Resumo}
\renewcommand{\contentsname}{índice Analítico}
\renewcommand{\refname}{Referências}
\renewcommand{\appendixname}{Apêndice}
\renewcommand{\tablename}{Tabela}

%% Formatação do cabeçalho e rodapé
\lhead{\footnotesize \Title} %
\rhead{\footnotesize \emph{\Author}} %
\chead{} %
\cfoot{} %
\lfoot{\footnotesize\nouppercase\leftmark} %
\rfoot{\bfseries\thepage}
\renewcommand{\footrulewidth}{0.1pt}

% Comando para inserir n\'{u}mero de documento
\newcounter{document}%[section]
\setcounter{document}{0}
\renewcommand\theenumi{\arabic{section}.\arabic{enumi}}
\newcommand\Doc{{\addtocounter{document}{1}\mbox{\sffamily\bfseries [Doc. \arabic{document}]}}}

% Comando para repetir um n\'{u}mero de documento j\'{a} citado
% \mbox{\sffamily{\bfseries{[Doc. XX]}}}

%% Alternativa na edição dos comandos.
%% Comando para inserir n\'{u}mero de documento
% \newcommand\thedocument{%
%    \ifthenelse{\arabic{subsection}=0}
%      {\thesection.\arabic{document}}
%      {\thesubsection.\arabic{document}}}
% \newcounter{document}[section]
% \setcounter{document}{0}
% \renewcommand\theenumi{\arabic{section}.\arabic{enumi}}
% \ifthenelse{\arabic{subsection} = 0}{\newcommand\Doc{{\stepcounter{document}\bfseries [Doc. \arabic{section}.\arabic{document}]}}}{\newcommand\Doc{{\stepcounter{document}\bfseries [Doc. \arabic{section}.\arabic{subsection}.\arabic{document}]}}}
% %\newcommand\Doc{{\addtocounter{document}{1}\mbox{\bfseries [Doc. \arabic{document}]}}}


% Ambiente para centralizar vertical
\newenvironment{vcenterpage}
     {\newpage\vspace*{\fill}}
     {\vspace*{\fill}\par\pagebreak}

\sloppy

\pagestyle{fancy}

\setcounter{secnumdepth}{4}

\begin{document}

\begin{titlepage}

\vspace{-5.0cm}


\begin{center}
\vspace{1cm}

\begin{figure}[!htb]
 \centering{\includegraphics[scale=0.7]{logo.png}}
 \label{fig:Instituicao_logo}
\end{figure}


%{\huge \textsf{Solicitação de Progressão Funcional Docente}} \\[1cm]

\rule{1.0\textwidth}{1pt} \\ [0.5cm]
{\Huge \textbf{\textsf{\Title}}} \\ 
\rule{1.0\textwidth}{1pt} \\ 



\doublespacing
\vspace{2cm}
{\Large \textsf{\Subject}}\\
\vspace{2cm}



{\LARGE \textsf{Solicitante: \textbf{\Author}}}\\


\vspace{2.0cm}
\doublespacing


\normalsize \textsf{\today}

\end{center}
\thispagestyle{empty}
\end{titlepage}



  \tableofcontents

%\include{Lista_Anexos} \cleartooddpage[\thispagestyle{empty}]

%%%%%%%%%%%%%%%%%%%%%%%%%%%%%%%%%%%%%%%%%%%%%%%%%%%%%%%%%%%%%%%%%%%%%%%%%%%%%%%
% APRESENTAção
%%%%%%%%%%%%%%%%%%%%%%%%%%%%%%%%%%%%%%%%%%%%%%%%%%%%%%%%%%%%%%%%%%%%%%%%%%%%%%%

\newpage
\section{Dados Pessoais}

\doublespacing
\noindent\textbf{Nome:} \Author \\
\textbf{Endereço:} Endereço completo  \\
\textbf{CPF:} numero de cpf do candidato  \\
\textbf{RG:} documento de identificação do candidato \\
\textbf{Telefones de contato:} números de telefone do candidato\\
\textbf{E-mail:} endereço de e-mail para contato \\
\textbf{Currículo Lattes:} link do Lattes




\subsection{Esclarecimentos}

Informações adicionais que o candidato julgar necessário



\section{Formação Acadêmica}
\singlespacing

\begin{enumerate}
\renewcommand{\labelenumi}{{\large\bfseries\arabic{enumi}.}}

    \item \textbf{Pós-doutorado }(anexo: comprovante n\degree \ref{reviewer:2015-sbcars} )
    \item \textbf{ Doutorado na área do concurso - título (ex: Doutorado em....; citar curso e área de concentração)  }(anexo: comprovante  n\degree \ref{reviewer:2015-sbcars})
       \Instituicao Nome Instituição
        \Conclusao 2021
        \RegDiploma 2002
    \item \textbf{Doutorado em outra área - título (ex: Doutorado em....; citar curso e área de concentração)} (anexo: comprovante  n\degree \ref{reviewer:2015-sbcars})
       \SubItemB{Instituição: }{Nome Instituição} 
        \SubItemB{Ano de conclusão : }{xxxx}
        \SubItemB{N\degree de Registro do Diploma}{ nro}
    \item \textbf{Mestrado na área do concurso - título (ex: Mestrado em....; citar curso e área de concentração)} (anexo: comprovante  n\degree \ref{reviewer:2015-sbcars}) 
       \SubItemB[Instituição: ]{Nome Instituição} 
        \SubItemB[Ano de conclusão : ]{xxxx}
        \SubItemB[N\degree de Registro do Diploma]{ nro}
    \item \textbf{Mestrado em outra área - título (ex: Mestrado em....; citar curso e área de concentração) }(anexo: comprovante  n\degree \ref{reviewer:2015-sbcars})
       \SubItemB{Instituição: }{Nome Instituição} 
        \SubItemB{Ano de conclusão : }{xxxx}
        \SubItemB{N\degree de Registro do Diploma}{ nro}
    \item \textbf{Especialização na área do concurso - título (ex: Bacharel em Direito, Ciência da Computação, Letras, etc.; Engenheiro Civil, 
Mecânico; etc.; Tecnólogo em Processamento de Dados, Construção Civil, etc) } (anexo: comprovante  n\degree \ref{reviewer:2015-sbcars})
       \SubItemB{Instituição: }{Nome Instituição} 
        \SubItemB{Ano de conclusão : }{xxxx}
        \SubItemB{N\degree de Registro do Diploma}{ nro} 
    \item \textbf{Especialização em outra área - título (ex: Bacharel em Direito, Ciência da Computação, Letras, etc.; Engenheiro Civil, 
Mecânico; etc.; Tecnólogo em Processamento de Dados, Construção Civil, etc) }(anexo: comprovante  n\degree \ref{reviewer:2015-sbcars})
        \SubItemB{Instituição: }{Nome Instituição} 
        \SubItemB{Ano de conclusão : }{xxxx}
        \SubItemB{N\degree de Registro do Diploma}{ nro}
    \item \textbf{Graduação na área do concurso- título (ex: Bacharel em Direito, Ciência da Computação, Letras, etc.; Engenheiro Civil, 
Mecânico; etc.; Tecnólogo em Processamento de Dados, Construção Civil, etc) } (anexo: comprovante  n\degree \ref{reviewer:2015-sbcars})
        \SubItemB{Instituição: }{Nome Instituição} 
        \SubItemB{Ano de conclusão : }{xxxx}
        \SubItemB{N\degree de Registro do Diploma}{ nro}
\end{enumerate}

\subsection{Formação Complementar}
%CURSOS DE EXTENSÃO, APERFEIÇOAMENTO, ESPECIALIZAÇÃO OU  EQUIVALENTES, NÃO INCLUÍDOS NOS TÍTULO DE GRADUAÇÃO, MESTRADO OU DOUTORADO.
\begin{enumerate}
\renewcommand{\labelenumi}{{\large\bfseries\arabic{enumi}.}}
    \item \textbf{nome do curso} (anexo: comprovante n\degree \ref{reviewer:2015-sbcars})
        \Instituicao nome 
        \Duracao num horas 
        \SubItemB nota de aproveitamento 
        \Periodo{inicio}{fim} 
\end{enumerate}

%%%%%%%%%%%%%%%%%%%%%%%%%%%%%%%%%%%%%%%%%%%%%%%%%%%%%%%%%%%%%%%%%%%%%%%%%%%%%%%
% Grupo 1 - Atividades de Ensino
%%%%%%%%%%%%%%%%%%%%%%%%%%%%%%%%%%%%%%%%%%%%%%%%%%%%%%%%%%%%%%%%%%%%%%%%%%%%%%%
\newpage
\section{Atividades de Ensino}
\vspace{0.3cm}

\begin{enumerate}
    \item \textbf{Instituição: } (anexo: comprovante n\degree \ref{reviewer:2015-sbcars} )
        \SubItemB Disciplinas Ensino Superior:
        \SubItemB Disciplinas Ensino Médio:
        \SubItemB Período:  
\end{enumerate}


%A seguir, listo as atividades de ensino que realizei no período, separadas por subgrupo, conforme rege o documento tal.

%%%%%%%%%%%%%%%%%%%%%%%%%%%%%%%%%%%%%%%%%%%%%%%%%%%%%%%%%%%%%%%%%%%%%%%%%%%%%%%
% Subgrupo 1.1 - Orientações e Co-Orientações
%%%%%%%%%%%%%%%%%%%%%%%%%%%%%%%%%%%%%%%%%%%%%%%%%%%%%%%%%%%%%%%%%%%%%%%%%%%%%%%
\subsection{Orientações e Co-Orientações}
\vspace{0.3cm}

\subsubsection{orientação de Teses de Doutorado Concluídas}
\vspace{0.3cm}

\begin{enumerate}
\renewcommand{\labelenumi}{{\large\bfseries\arabic{enumi}.}}

\item       \textbf{Aluno:} Jack Allan Bauer \mbox{\sffamily{\bfseries{[Doc. \ref{advisor:phd-concluidas}]}}} \\
            \textbf{Título da Tese:} Salvando o mundo em 24 horas.\\
            \textbf{Data da Defesa:} 6 de Novembro de 2001 \\
            \textbf{Instituição:} Universidade Federal de Pernambuco.

\end{enumerate}

%------------------------------------------------------------------------------

\subsubsection{Co-orientação de Teses de Doutorado Concluídas}
\vspace{0.3cm}

\begin{enumerate}
\renewcommand{\labelenumi}{{\large\bfseries\arabic{enumi}.}}

\item       \textbf{Aluno:} Thomas A. Anderson [Doc. \ref{co-advisor:phd-concluidas}]\\
            \textbf{Título da Tese:} The Matrix \\
            \textbf{Data da Defesa:} 31 de Março de 1999 \\
            \textbf{Instituição:} Universidade Federal de Pernambuco.

\end{enumerate}

%------------------------------------------------------------------------------

\subsubsection{orientação de Teses de Doutorado em Andamento}
\vspace{0.3cm}

\begin{enumerate}
\renewcommand{\labelenumi}{{\large\bfseries\arabic{enumi}.}}

\item       \textbf{Aluno:} Morpheus Laurence Fishburne) [Doc. \ref{advisor:phd-andamento}] \\
            \textbf{Título da Tese:} Nabucodonozor in The Matrix \\
            \textbf{Data de Início:} Março de 2001 \\
            \textbf{Instituição:} CIn/UFPE

\end{enumerate}

%------------------------------------------------------------------------------

\subsubsection{Co-orientação de Teses de Doutorado em Andamento}
\vspace{0.3cm}

\begin{enumerate}
\renewcommand{\labelenumi}{{\large\bfseries\arabic{enumi}.}}

\item       \textbf{Aluno:} Anakin Skywalker [Doc. \ref{co-advisor:phd-andamento}]\\
            \textbf{Título da Tese:} A Ameaça Fantasma \\
            \textbf{Data de Início:} Maio de 1999 \\
            \textbf{Instituição:} CIn/UFPE

\end{enumerate}

%------------------------------------------------------------------------------

\subsubsection{orientação de Dissertações de Mestrado Concluídas}
\vspace{0.3cm}

\begin{enumerate}
\renewcommand{\labelenumi}{{\large\bfseries\arabic{enumi}.}}

\item       \textbf{Aluno:}  Obi-Wan Kenobi [Doc. \ref{advisor:msc-concluidas}] \\
            \textbf{Título da Dissertação:} A Vingança dos Sith\\
            \textbf{Tipo:} Acadêmico \\%| Profissional\\
            \textbf{Data da Defesa:} 19 de Maio de 2005\\
            \textbf{Instituição:} CIn/UFPE.

\item       \textbf{Aluno:} Darth Maul [Doc. \ref{advisor:mprof-concluidas}] \\
            \textbf{Título da Dissertação:} Lord of the Sith\\
            \textbf{Tipo:} Profissional\\
            \textbf{Data da Defesa:} 19 de Maio de 2005\\
            \textbf{Instituição:} CIn/UFPE.

\end{enumerate}

%------------------------------------------------------------------------------

\subsubsection{Co-orientação de Dissertações de Mestrado Concluídas}
\vspace{0.3cm}

\begin{enumerate}
\renewcommand{\labelenumi}{{\large\bfseries\arabic{enumi}.}}

\item       \textbf{Aluno:} Anthony Edward Stark [Doc. \ref{co-advisor:msc-concluidas}] \\
            \textbf{Título da Dissertação:} Tales of Suspense\\
            \textbf{Tipo:} Acadêmico \\%| Profissional\\
            \textbf{Data da Defesa:} Março de 1963 \\
            \textbf{Instituição:} Marvel Comics

\item       \textbf{Aluno:} Henry Jonathan Pym [Doc. \ref{co-advisor:mprof-concluidas}] \\
            \textbf{Título da Dissertação:} Homem-Formiga\\
            \textbf{Tipo:} Profissional\\
            \textbf{Data da Defesa:} Março de 1963 \\
            \textbf{Instituição:} Marvel Comics

\end{enumerate}

%------------------------------------------------------------------------------

\subsubsection{orientação de Dissertações de Mestrado em Andamento}
\vspace{0.3cm}

\begin{enumerate}
\renewcommand{\labelenumi}{{\large\bfseries\arabic{enumi}.}}

\item       \textbf{Aluno:} Stanley Martin Lieber [Doc. \ref{advisor:msc-andamento}]\\
            \textbf{Título da Dissertação:} Stan Lee\\
            \textbf{Tipo:} Acadêmico \\%| Profissional\\
            \textbf{Data de Início:} 22 de Dezembro de 1922\\
            \textbf{Instituição:} Marvel Comics

\item       \textbf{Aluno:} Robert Bruce Banner [Doc. \ref{advisor:mprof-andamento}]\\
            \textbf{Título da Dissertação:} Avengers vol. 1\\
            \textbf{Tipo:} Profissional\\
            \textbf{Data de Início:} 22 de Dezembro de 1922\\
            \textbf{Instituição:} Marvel Comics
\end{enumerate}

%------------------------------------------------------------------------------
\subsubsection{Co-orientação de Dissertações de Mestrado em Andamento}
\vspace{0.3cm}

\begin{enumerate}
\renewcommand{\labelenumi}{{\large\bfseries\arabic{enumi}.}}

\item       \textbf{Aluno:} Erik Magnus Lehnsherr [Doc. \ref{co-advisor:msc-andamento}]\\
            \textbf{Título da Dissertação:} Magneto\\
            \textbf{Tipo:} Acadêmico \\%| Profissional\\
            \textbf{Data de Início:} Setembro de 1963\\
            \textbf{Instituição:} Marvel Comics

\item       \textbf{Aluno:} Charles Francis Xavier [Doc. \ref{co-advisor:mprof-andamento}]\\
            \textbf{Título da Dissertação:} Magneto\\
            \textbf{Tipo:} Profissional\\
            \textbf{Data de Início:} Setembro de 1963\\
            \textbf{Instituição:} Marvel Comics
\end{enumerate}

%------------------------------------------------------------------------------

\subsubsection{orientação de Trabalhos de Conclusão de Curso}
\vspace{0.3cm}

\begin{enumerate}
\renewcommand{\labelenumi}{{\large\bfseries\arabic{enumi}.}}

\item       \textbf{Aluno:} Uhtred Ragnarson [Doc. \ref{advisor:tcc}]\\
            \textbf{Curso:} Crônicas Saxônicas\\
            \textbf{Título da Monografia:} O Último Reino\\
            \textbf{Data da Defesa:} Maio de 2004\\
            \textbf{Instituição:} Bebbanburg

\end{enumerate}

%------------------------------------------------------------------------------

\subsubsection{orientação de Monitoria}
\vspace{0.3cm}

\begin{enumerate}
\renewcommand{\labelenumi}{{\large\bfseries\arabic{enumi}.}}

\item   \textbf{Aluno:} Ragnar Lodbrok [Doc. \ref{advisor:monitoria}]\\
        \textbf{Disciplina:}  Engenharia Software e Sistemas (turma SI)\\
        \textbf{Curso:} Sistemas de Informação\\
        \textbf{Semestre:} 2015.2

\end{enumerate}

%------------------------------------------------------------------------------

\subsubsection{orientação de Trabalhos de Iniciação Científica}
\vspace{0.3cm}

\begin{enumerate}
\renewcommand{\labelenumi}{{\large\bfseries\arabic{enumi}.}}

\item   \textbf{Aluno:} Gorm den Gamle [Doc. \ref{advisor:icc}] \\
        \textbf{Projeto:} Pedras de Jelling \\
        \textbf{Tema:} Nórdico antigo \\
        \textbf{Período:} 936 - 938\\
        \textbf{Financiamento:} CAPES

\end{enumerate}

%------------------------------------------------------------------------------

\subsubsection{orientação de Trabalhos de Apoio Acadêmico}
\vspace{0.3cm}

\begin{enumerate}
\renewcommand{\labelenumi}{{\large\bfseries\arabic{enumi}.}}

\item   \textbf{Aluno:} Qui-Gon [Doc. \ref{advisor:bia}]\\
        \textbf{Projeto:} Uso da Internet das Coisas no acesso multimodal a informações em um Smart Campus.\\
        \textbf{Tema:} Comunicações sem Fio \\
        \textbf{Categoria:} Iniciação acadêmica (Graduando em Ciência da Computação) – Universidade Federal de Pernambuco.\\
        \textbf{Período:} 01 de Outubro de 2014 a 30 de Setembro de 2015 \\
        \textbf{Financiamento:} Programa Institucional de Bolsa de Incentivo Acadêmico – BIA, Fundação de Amparo à Ciência e Tecnologia do Estado de Pernambuco (FACEPE), processo número: BIA-0155-1.03/14.

\end{enumerate}

%%%%%%%%%%%%%%%%%%%%%%%%%%%%%%%%%%%%%%%%%%%%%%%%%%%%%%%%%%%%%%%%%%%%%%%%%%%%%%%
% Subgrupo 1.2 - participação em Comissões Examinadoras
%%%%%%%%%%%%%%%%%%%%%%%%%%%%%%%%%%%%%%%%%%%%%%%%%%%%%%%%%%%%%%%%%%%%%%%%%%%%%%%
\subsection{participação em Comissões Examinadoras}
\vspace{0.3cm}

\subsubsection{Bancas Examinadoras de Concurso}
\vspace{0.3cm}

\begin{enumerate}
\renewcommand{\labelenumi}{{\large\bfseries\arabic{enumi}.}}
\vspace{0.3cm}

\item       \textbf{Descrição:} Parecerista do(s) curso(s) de Sistemas de Informação da Avaliação de Cursos Superiores do Guia do Estudante (GE) 2015 [Doc. \ref{app:2015-guia-estudante}]\\
            \textbf{Instituição:} Guia do Estudante / Editora Abril \\
            \textbf{Data/Período:} 2015 %16 de Agosto de 2015

\end{enumerate}

%------------------------------------------------------------------------------

\subsubsection{Bancas Congressos de Iniciação Científica ou de Extensão}
\vspace{0.3cm}

\begin{enumerate}
\renewcommand{\labelenumi}{{\large\bfseries\arabic{enumi}.}}
\vspace{0.3cm}

\item   \textbf{Descrição:}  Participação na Comissão Avaliadora dos trabalhos na área de Ciências Exatas da 19ª Jornada de Iniciação Científica da FACEPE [Doc. \ref{app:2015-jic-facepe}]
    \SubItemB  Instituição: Fundação de Amparo a Ciência e Tecnologia do Estado de Pernambuco 
     \SubItemB Data/Período: 09 a 12 de Junho de 2015
\end{enumerate}

%------------------------------------------------------------------------------

\subsubsection{Bancas de Trabalho de Conclusão de Curso}
\vspace{0.3cm}

\begin{enumerate}
\renewcommand{\labelenumi}{{\large\bfseries\arabic{enumi}.}}
\vspace{0.3cm}

\item       \textbf{Aluno:} José Welington de Almeida Filho [Doc. \ref{app:bancas-tcc}]
    \SubItemBB  Título da Monografia: Uma proposta de uso de Gamification para o ensino de software.
    \SubItemB  Curso: Graduação em Ciência da Computação\\
           \SubItemB Universidade: Universidade Federal de Pernambuco (UFPE)
           \SubItemB  Data: 28 de Julho de 2015\\

\end{enumerate}

%------------------------------------------------------------------------------

\subsubsection{Bancas de Dissertação de Mestrado}
\vspace{0.3cm}

\begin{enumerate}
\renewcommand{\labelenumi}{{\large\bfseries\arabic{enumi}.}}
\vspace{0.3cm}

\item       \textbf{Candidato:} Helaine Solange Lins Barreiros [Doc. \ref{app:2015-msc-hslb}] \\
            \textbf{Título da Dissertação:} Análise da Completude dos Relatos de Experimentos em Elasticidade na Computação em Nuvem: Um Mapeamento Sistemático\\
            \textbf{Tipo:} Acadêmico\\
            \textbf{Programa:} Programa de Pós-Graduação em Ciência da Computação\\
            \textbf{Universidade:} Universidade Federal de Pernambuco (UFPE)\\
            \textbf{Data:} 31 de Agosto de 2015

\end{enumerate}

%------------------------------------------------------------------------------

\subsubsection{Bancas de Qualificaç\~ao / Proposta de Tese de Doutorado}
\vspace{0.3cm}

\begin{enumerate}
\renewcommand{\labelenumi}{{\large\bfseries\arabic{enumi}.}}
\vspace{0.3cm}

\item       \textbf{Candidato:} Thiago Monteiro Prota [Doc. \ref{app:2015-quali-phd-tmp}] \\
            \textbf{Título da Qualificação:} Análises Estrutural e Comportamental Orientadas a Conformidade para o Desenvolvimento de Aplicações Multimídia\\
            \textbf{Programa:} Programa de Pós-Graduação em Ciência da Computação\\
            \textbf{Universidade:} Universidade Federal de Pernambuco (UFPE)\\
            \textbf{Data:} 18 de Setembro de 2015

\end{enumerate}



%------------------------------------------------------------------------------

\subsubsection{participação em Bancas de Seleção para ingresso e exames de Qualificaç\~ao de Programa de Pó-Graduação \textit{Stricto Sensu}}

\begin{enumerate}
\renewcommand{\labelenumi}{{\large\bfseries\arabic{enumi}.}}
\vspace{0.3cm}

\item       \textbf{Descrição:} Membro Externo da Comissão de Seleção para o Doutorado Sanduíche - PDSE da CAPES [Doc. \ref{app:2015-pgcomp-pdse}] \\
            \textbf{Programa:} Programa de Pós-Graduação em Ciência da Computação (PGCOMP) do aluno Alcemir Rodrigues Santos\\
            \textbf{Instituição:} Universidade Federal da Bahia (UFBA) \\
            \textbf{Período de Realização:} 15 de Junho de 2015

\end{enumerate}

%%%%%%%%%%%%%%%%%%%%%%%%%%%%%%%%%%%%%%%%%%%%%%%%%%%%%%%%%%%%%%%%%%%%%%%%%%%%%%%
% Subgrupo 1.3 - Atividades de Ensino na Graduação e na Pós-Graduação
%%%%%%%%%%%%%%%%%%%%%%%%%%%%%%%%%%%%%%%%%%%%%%%%%%%%%%%%%%%%%%%%%%%%%%%%%%%%%%%





%%%%%%%%%%%%%%%%%%%%%%%%%%%%%%%%%%%%%%%%%%%%%%%%%%%%%%%%%%%%%%%%%%%%%%%%%%%%%%%
% Grupo 2: Atividades de Produção Científica e Técnica, Artística e Cultural
%%%%%%%%%%%%%%%%%%%%%%%%%%%%%%%%%%%%%%%%%%%%%%%%%%%%%%%%%%%%%%%%%%%%%%%%%%%%%%%
\newpage
\section{Atividades de Produção Científica e Técnica, Artística e Cultural}

%%%%%%%%%%%%%%%%%%%%%%%%%%%%%%%%%%%%%%%%%%%%%%%%%%%%%%%%%%%%%%%%%%%%%%%%%%%%%%%
% Subgrupo 2.1 - Produtividade de Pesquisa
%%%%%%%%%%%%%%%%%%%%%%%%%%%%%%%%%%%%%%%%%%%%%%%%%%%%%%%%%%%%%%%%%%%%%%%%%%%%%%%
\subsection{Produtividade de Pesquisa}
\vspace{0.3cm}

%------------------------------------------------------------------------------

\subsubsection{Bolsista de produtividade em pesquisa e em inovação tecnológica}
\vspace{0.3cm}

\begin{enumerate}
\renewcommand{\labelenumi}{{\large\bfseries\arabic{enumi}.}}

\item \textbf{Título do projeto:} The Death Star \textbf{[Doc. \ref{app:bolsista-prod}]}\\
      \textbf{N\'{u}mero do processo:} DV-1123-5.8/13\\
      \textbf{Financiador/Edital:} Conselho Nacional de Desenvolvimento Científico e Tecnológico, Edital MCTI/CNPq Nº 14/2013\\
      \textbf{Período (início-fim):} 2015 - 2018\\
      \textbf{Situação:} Desenvolvimento tecnológico\\
      \textbf{Natureza:} Pesquisa.

\end{enumerate}

%------------------------------------------------------------------------------

\subsubsection{participação em Eventos Científicos (com apresentação de trabalho ou oferecimento de cursos, palestras ou debates)}
\vspace{0.3cm}

\begin{enumerate}
\renewcommand{\labelenumi}{{\large\bfseries\arabic{enumi}.}}

\item   \textbf{Evento:} Congresso Brasileiro de Software: Teoria e Prática [Doc. \ref{app:2015-cbsoft}] \\
        \textbf{Propósito:} Participante\\
        \textbf{Período:} 21 a 25 de Setembro de 2015\\
        \textbf{Local:} Belo Horizonte-MG, Brasil.

\item   \textbf{Evento:} XII Encontro Anual de Computação (EnAComp) [Doc. \ref{app:2015-enacomp}] \\
        \textbf{Propósito (i):} Ouvinte\\
        \textbf{Propósito (ii):} Apresentação da palestra ``Empreendedorismo e criação de negócios intensivos em software''\\
        \textbf{Propósito (ii):} Apresentação do minicurso ``Desenvolvimento de SaaS / Engenharia de Sofwtarte como Serviço'', com duração de 12 horas\\
        \textbf{Período:} 16 a 18 de Setembro de 2015\\
        \textbf{Local:} , Brasil.

\end{enumerate}

%------------------------------------------------------------------------------
% TODO
\subsubsection{Autoria de artigos completos publicados em anais de congresso, em jornais e revistas de circulação nacional e internacional na sua área}
\vspace{0.3cm}

\begin{enumerate}
\renewcommand{\labelenumi}{{\large\bfseries\arabic{enumi}.}}

\item GAMA, Kiev Santos; WANDERLEY, Rafael Roballo; MARANHÃO, Daniel Barbosa; GARCIA, Vinicius Cardoso. \textbf{TagHunt: Uma plataforma Combinando a Internet das Coisas com Scavenger Hunt Games}. In: 7º Simpósio Brasileiro de Computação Ubíqua e Pervasiva (SBCUP), Evento do XXXV Congresso da Sociedade Brasileira de Computação, Recife, 20 a 23 de Julho de 2015. \textbf{[Doc. \ref{conf:2015-sbcup}]}

\end{enumerate}

%------------------------------------------------------------------------------
% TODO
\subsubsection{Arbitragem de Artigos Técnico-Científicos Nacionais e Internacionais na sua área de atuação}
\vspace{0.3cm}

\begin{enumerate}
\renewcommand{\labelenumi}{{\large\bfseries\arabic{enumi}.}}

\item   \textbf{Periódico:} IEEE Transactions on Services Computing \textbf{[Doc. \ref{reviewer:2015-tsc}]}\\
        \textbf{Editora:} IEEE Computer Society\\
        \textbf{ISSN:} 1939-1374\\
        \textbf{URL:} \url{http://www.computer.org/web/tsc}

\end{enumerate}

%------------------------------------------------------------------------------
% TODO
\subsubsection{Coordenação e/ou participação em Projetos Aprovados por órgãos de Fomento}
\vspace{0.3cm}

\begin{enumerate}
\renewcommand{\labelenumi}{{\large\bfseries\arabic{enumi}.}}

\item \textbf{Título do projeto:} BIGStore - Evolução da plataforma Ustore para Armazenamento, Manipulação e Experimentação de Grandes Volumes de Dados \textbf{[Doc. \ref{project:2015-facepe-pepe}]}\\
      \textbf{Função no projeto:} Integrante\\
      \textbf{N\'{u}mero do processo:} SIN-0199-1.03/15\\
      \textbf{Financiador/Edital:} Fundação de Amparo à Ciência e Tecnologia do Estado de Pernambuco, Edital FACEPE 23/2014 - PESQUISADOR NA EMPRESA DE PERNAMBUCO (PEPE)\\
      \textbf{Período (início-fim):} 2015 - 2018\\
      \textbf{Situação:} Desenvolvimento tecnológico\\
      \textbf{Natureza:} Pesquisa.

\end{enumerate}

%------------------------------------------------------------------------------

\subsubsection{Consultoria às Instituições de Fomento à Pesquisa, Ensino e Extensão}
\vspace{0.3cm}

\begin{enumerate}
\renewcommand{\labelenumi}{{\large\bfseries\arabic{enumi}.}}

\item   \textbf{Função:} Avaliador de Projetos de Pesquisa (modalidade Subvenção Econ\^{o}mica - PAPPE Integração) - Edital 10.2/2012 \textbf{[Doc. \ref{consulting:2015-facepe-pepe}]}\\
        \textbf{Instituição:} Fundação de Amparo à Ciência e Tecnologia do Estado de Pernambuco (FACEPE).

\end{enumerate}

%------------------------------------------------------------------------------

\subsubsection{Prêmios Recebidos pela Produção Científica e Técnica}
\vspace{0.3cm}

Nada a declarar neste subgrupo.

\begin{enumerate}
\renewcommand{\labelenumi}{{\large\bfseries\arabic{enumi}.}}

\item Melhor artigo publicado no Simpósio Intergálito da República, 2014 \textbf{[Doc. \ref{award:2014}]}.

\end{enumerate}

%%%%%%%%%%%%%%%%%%%%%%%%%%%%%%%%%%%%%%%%%%%%%%%%%%%%%%%%%%%%%%%%%%%%%%%%%%%%%%%
% Subgrupo 2.2 - Produção Científica
%%%%%%%%%%%%%%%%%%%%%%%%%%%%%%%%%%%%%%%%%%%%%%%%%%%%%%%%%%%%%%%%%%%%%%%%%%%%%%%

\subsection{Produção Científica}
\vspace{0.3cm}

%------------------------------------------------------------------------------

\subsubsection{Trabalhos Publicados em Periódicos Especializados do País ou do Exterior}
\vspace{0.3cm}

\begin{enumerate}
\renewcommand{\labelenumi}{{\large\bfseries\arabic{enumi}.}}

\item AFONSO, Ricardo Alexandre ; BRITO, Kellyton dos Santos ; DO NASCIMENTO, CLÓVIS HOLANDA ; COSTA, L.  C. ; ÁLVARO, ALEXANDRE ; Garcia, Vinicius Cardoso . ``(Br-SCMM) Brazilian Smart City Maturity Model: A Perspective from the Health Domain''. Studies in Health Technology and Informatics, v. 216 (MEDINFO 2015: eHealth-enabled Health), p. 983, 2015, doi: 10.3233/978-1-61499-564-7-983 \colorbox{yellow}{(Qualis XX)} \textbf{[Doc. \ref{journal:2015-medinfo}]}

\end{enumerate}

%%%%%%%%%%%%%%%%%%%%%%%%%%%%%%%%%%%%%%%%%%%%%%%%%%%%%%%%%%%%%%%%%%%%%%%%%%%%%%%
% Grupo 3: Atividades de Extensão
%%%%%%%%%%%%%%%%%%%%%%%%%%%%%%%%%%%%%%%%%%%%%%%%%%%%%%%%%%%%%%%%%%%%%%%%%%%%%%%
\newpage
\section{Atividades de Extensão}

%%%%%%%%%%%%%%%%%%%%%%%%%%%%%%%%%%%%%%%%%%%%%%%%%%%%%%%%%%%%%%%%%%%%%%%%%%%%%%%
% Subgrupo 3.1 - Coordenação e Orientação
%%%%%%%%%%%%%%%%%%%%%%%%%%%%%%%%%%%%%%%%%%%%%%%%%%%%%%%%%%%%%%%%%%%%%%%%%%%%%%%
\subsection{Subgrupo 3.1 - Coordenação e Orientação}
\vspace{0.3cm}

\begin{enumerate}
\renewcommand{\labelenumi}{{\large\bfseries\arabic{enumi}.}}

    \item Coordenação de Projeto de Extensão \textbf{[Doc. \ref{extensao:2015}]}

\end{enumerate}

%%%%%%%%%%%%%%%%%%%%%%%%%%%%%%%%%%%%%%%%%%%%%%%%%%%%%%%%%%%%%%%%%%%%%%%%%%%%%%%
% Subgrupo 3.2 - Coordenação de Eventos e Conferencista
%%%%%%%%%%%%%%%%%%%%%%%%%%%%%%%%%%%%%%%%%%%%%%%%%%%%%%%%%%%%%%%%%%%%%%%%%%%%%%%
\subsection{Coordenação de Eventos e Conferencista}
\vspace{0.3cm}

%------------------------------------------------------------------------------

\subsubsection{Comissão Organizadora de Eventos Internacional, Nacional, Regional ou Local}
\vspace{0.3cm}

\begin{enumerate}
\renewcommand{\labelenumi}{{\large\bfseries\arabic{enumi}.}}

    \item Program Committee Member of the 9th Brazilian Symposium on Software Components, Architectures and Reuse (SBCARS 2015) \textbf{[Doc. \ref{reviewer:2015-sbcars}]}

\end{enumerate}





%%%%%%%%%%%%%%%%%%%%%%%%%%%%%%%%%%%%%%%%%%%%%%%%%%%%%%%%%%%%%%%%%%%%%%%%%%%%%%%
% LISTA DE ANEXOS
%%%%%%%%%%%%%%%%%%%%%%%%%%%%%%%%%%%%%%%%%%%%%%%%%%%%%%%%%%%%%%%%%%%%%%%%%%%%%%%

\newpage
\section{Lista de Anexos}

A Tabela \ref{Tab:ListaAnexos} contém a numeração e uma pequena descrição dos documentos comprobatórios em anexo. Quando for o caso, há uma indicação entre parênteses, ao final da descrição, do setor responsável pela emissão do documento.


\begin{table}[ht]
\small
\caption{\texttt{Relação numerada dos Anexos comprobatórios}.}
\begin{tabular}{cl}
\toprule
\large{\textbf{\texttt{Doc}}} & \multicolumn{1}{c}{\large{\textbf{\texttt{descrição}}}} \\
\otoprule
  1 & Declaração de orientação de Mestrado de Maxwell Silva (PPGES/UPE). \\
  %\cmidrule{1-2}
  2 & Declaração de orientação de Mestrado de Bruna Melo (PPGES/UPE). \\
  %\cmidrule{1-2}
  3 & Declaração de orientação de TCC de  (Graduação EC CIn). \\
  %\cmidrule{1-2}
  4 & Termo de orientação de Monitoria de  2013.2 (PROACAD/UFPE). \\
  %\cmidrule{1-2}
  5 & Declaração de orientação de Monitoria de- 2013.1 (Sec Grad CIn). \\
  %\cmidrule{1-2}
  6 & Declaração de orientação de IC  \\
  %\cmidrule{1-2}
  7 & Declaração de participação em banca de Doutorado  (PPGEE/UFPE). \\
  %\cmidrule{1-2}
  8 & Declaração de participação em banca de Doutorado de  (PPGEE/UFPE). \\
  \cmidrule{1-2}
  9 & Declaração de participação em banca de Mestrado de  (PGCC/UFPE). \\
    & Cópia da Portaria PGCC 166/2012 -Composição de banca de Mestrado (PGCC/UFPE). \\
  \cmidrule{1-2}
  10 & Declaração de participação em banca de Mestrado de Antonio Assunção (PPGES/UPE). \\
     & Cópia da Portaria PPGES 11/2013 -Composição de banca de Mestrado (PPGES/UPE). \\
  \cmidrule{1-2}
  11 & Declaração de participação em banca de Mestrado de Maxwell Silva (PPGES/UPE). \\
     & Cópia da Portaria PPGES 15/2013 -Composição de banca de Mestrado (PPGES/UPE). \\
  \cmidrule{1-2}
  12 & Declaração de participação em banca de Mestrado de Bruna Melo (PPGES/UPE). \\
     & Cópia da Portaria PPGES 16/2013 -Composição de banca de Mestrado (PPGES/UPE). \\
  \cmidrule{1-2}
  13 & Declaração de participação em banca de TCC (Graduação-EC CIn). \\
  %\cmidrule{1-2}
  14 & Declaração de particip. em banca de Seleção de Prof. Tempor\'{a}rio do CIn/UFPE (Direção CIn). \\
  %\cmidrule{1-2}
  15 & Relatório de disciplinas lecionadas na Graduação no período de avaliação (SIGA-UFPE).\\
  %\cmidrule{1-2}
  16 & Cópia do certificado de participação no XXX SBrT. \\
  %\cmidrule{1-2}
  17 & Cópia do certificado de participação no XXXI SBrT. \\
  %\cmidrule{1-2}
  18 & Cópias do resumo publicado e do e-mail de aceitação de trabalho no XXXI SBrT. \\
  %\cmidrule{1-2}
  19 & Cópias do artigo publicado e do e-mail de aceitação de trabalho no XXX SBrT. \\
  %\cmidrule{1-2}
  20 a 23 & Cópias dos artigos publicados e dos e-mails de aceitação de trabalho no CBA 2012 e XXXI SBrT. \\
  %\cmidrule{1-2}
  24 a 31 & Cópias dos e-mails de convite para revisão e finalização do processo (periódicos e eventos). \\
  \cmidrule{1-2}
  32 & Cópia do Resultado do Edital 10/2010 (pags 1,2 e 6) e Termo de Outorga do projeto (FACEPE). \\
     & Cópia de correspondência comprobatória de prestação de contas do projeto (FACEPE). \\
  \cmidrule{1-2}
  33 & Cópias de e-mail de convite para avaliador e de cabeçalho do parecer emitido (FACEPE). \\
  %\cmidrule{1-2}
  34 & Cópia do certificado de premiação de melhor artigo de IC do XXXI SBrT. \\
  %\cmidrule{1-2}
  35 & Cópias do artigo publicado no periódico TEMA e do e-mail de aceitação. \\
  %\cmidrule{1-2}
  36 & Cópias do artigo publicado no periódico EURASIP J. Adv. Sig. Proc. e do e-mail de aceitação. \\
  %\cmidrule{1-2}
  37 & Cópias dos e-mails comprobatórios de convite e resposta ao convite (MIC-CSC2012). \\
  %\cmidrule{1-2}
  38 & Cópias dos e-mails comprobatórios de convite e resposta ao convite (XXXI SBrT). \\
  %\cmidrule{1-2}
  39 & Cópia da Declaração enviada à Comissão Organizadora da SBPC 2013 (Direção CIn). \\
  %\cmidrule{1-2}
  40 & Online Course Statement of Accomplishment: Machine Learning - Stanford University - Coursera. \\
  %\cmidrule{1-2}
  41 - 42 & Declarações de participação na Comissão de Seleção do Mestrado CIn - 2012 e 2013. \\
  %\cmidrule{1-2}
  43 & Declaração comprobatória de cargo ocupado - Subcoordenação de Editais. \\
  %\cmidrule{1-2}
  44 & Declaração de participação no Colegiado da Pós-Graduação do CIn. \\
  %\cmidrule{1-2}
  45 & Cópia da Portaria de Designação 005/2013 - Colegiado Graduação EC CIn (Direção CIn). \\
\bottomrule
\end{tabular}
\label{Tab:ListaAnexos}
\end{table}

% Appendix
\clearpage
%\addappheadtotoc
\appendix
%\appendixpage
\newpage
\section{Documentos comprobatórios}
Esta seção contém os documentos comprobatórios referentes às atividades listadas neste memorial.
\addcontentsline{toc}{section}{Documentos comprobatórios}
\renewcommand{\thesubsection}{\arabic{subsection}}
% \renewcommand{\subsection}{
% \titleformat{\subsection}
%   {\Huge\bfseries\center\vspace{.4\textwidth}\thispagestyle{fancy}} % format
%   {}                % label
%   {0pt}             % sep
%   {\huge}           % before-code
% }

%%%%%%%%%%%%%%%%%%%%%%%%%%%%%%%%%%%%%%%%%%%%%%%%%%%%%%%%%%%%%%%%%%%%%%%%%%%%%%%
% Grupo 1 - Atividades de Ensino
%%%%%%%%%%%%%%%%%%%%%%%%%%%%%%%%%%%%%%%%%%%%%%%%%%%%%%%%%%%%%%%%%%%%%%%%%%%%%%%

%%%%%%%%%%%%%%%%%%%%%%%%%%%%%%%%%%%%%%%%%%%%%%%%%%%%%%%%%%%%%%%%%%%%%%%%%%%%%%%
% Subgrupo 1.1 - Formação Acadêmica
%%%%%%%%%%%%%%%%%%%%%%%%%%%%%%%%%%%%%%%%%%%%%%%%%%%%%%%%%%%%%%%%%%%%%%%%%%%%%%%

\newpage
\subsection{Formação Acadêmica}
\label{advisor:phd-concluidas}

\bigskip
\begin{minipage}{\textwidth}
\includepdf[pages=-, scale=1,pagecommand={}]{\detokenize{GRUPO 1/Sub-Grupo 11/Comprovante Fake}}
\end{minipage}
\newpage
\subsection{Co-Orientação de Teses de Doutorado Concluídas}
\label{co-advisor:phd-concluidas}
Esta subseção apresenta o comprovante de co-orientações de Tese de Doutorado concluídas.\\

\bigskip
\begin{minipage}{\textwidth}
\includepdf[pages=-, scale=1,pagecommand={}]{\detokenize{GRUPO 1/Sub-Grupo 11/Comprovante Fake}}
\end{minipage}

\newpage
\subsection{Orientação de Teses de Doutorado em Andamento}
\label{advisor:phd-andamento}

\bigskip
\begin{minipage}{\textwidth}
\includepdf[pages=-, scale=1,pagecommand={}]{\detokenize{GRUPO 1/Sub-Grupo 11/Comprovante Fake}}
\end{minipage}

\newpage
\subsection{Co-Orientação de Teses de Doutorado em Andamento}
\label{co-advisor:phd-andamento}

\bigskip
\begin{minipage}{\textwidth}
\includepdf[pages=-, scale=1,pagecommand={}]{\detokenize{GRUPO 1/Sub-Grupo 11/Comprovante Fake}}
\end{minipage}

\newpage
\subsection{Orientação de Dissertações de Mestrado Concluídas}
\label{advisor:msc-concluidas}

\bigskip
\begin{minipage}{\textwidth}
\includepdf[pages=-, scale=1,pagecommand={}]{\detokenize{GRUPO 1/Sub-Grupo 11/Comprovante Fake}}
\end{minipage}

\newpage
\subsection{Co-Orientação de Dissertações de Mestrado Concluídas}
\label{co-advisor:msc-concluidas}
Esta subseção apresenta o comprovante de co-orientações de Dissertação de Mestrado concluídas.\\

\bigskip
\begin{minipage}{\textwidth}
\includepdf[pages=-, scale=1,pagecommand={}]{\detokenize{GRUPO 1/Sub-Grupo 11/Comprovante Fake}}
\end{minipage}

\newpage
\subsection{Orientação de Dissertações de Mestrado em Andamento}
\label{advisor:msc-andamento}
Esta subseção apresenta o comprovante de orientações de Dissertação de Mestrado em andamento.\\

\bigskip
\begin{minipage}{\textwidth}
\includepdf[pages=-, scale=1,pagecommand={}]{\detokenize{GRUPO 1/Sub-Grupo 11/Comprovante Fake}}
\end{minipage}

\newpage
\subsection{Co-Orientação de Dissertações de Mestrado em Andamento}
\label{co-advisor:msc-andamento}
Esta subseção apresenta o comprovante de co-orientações de Dissertação de Mestrado em andamento.\\

\bigskip
\begin{minipage}{\textwidth}
\includepdf[pages=-, scale=1,pagecommand={}]{\detokenize{GRUPO 1/Sub-Grupo 11/Comprovante Fake}}
\end{minipage}

\newpage
\subsection{Orientação de Dissertação de Mestrado Profissional Conluídas}
\label{advisor:mprof-concluidas}
Esta subseção apresenta o comprovante de orientações de Dissertação de Mestrado Profissional concluídas.\\

\bigskip
\begin{minipage}{\textwidth}
\includepdf[pages=-, scale=1,pagecommand={}]{\detokenize{GRUPO 1/Sub-Grupo 11/Comprovante Fake}}
\end{minipage}

\newpage
\subsection{Co-Orientação de Dissertação de Mestrado Profissional Conluídas}
\label{co-advisor:mprof-concluidas}
Esta subseção apresenta o comprovante de co-orientações de Dissertação de Mestrado Profissional concluídas.\\

\bigskip
\begin{minipage}{\textwidth}
\includepdf[pages=-, scale=1,pagecommand={}]{\detokenize{GRUPO 1/Sub-Grupo 11/Comprovante Fake}}
\end{minipage}

\newpage
\subsection{Orientações de Mestrado Profissional em Andamento}
\label{advisor:mprof-andamento}
Esta subseção apresenta o comprovante de orientações de Dissertação de Mestrado Profissional em andamento.\\

\bigskip
\begin{minipage}{\textwidth}
\includepdf[pages=-, scale=1,pagecommand={}]{\detokenize{GRUPO 1/Sub-Grupo 11/Comprovante Fake}}
\end{minipage}

\newpage
\subsection{Co-Orientações de Mestrado Profissional em Andamento}
\label{co-advisor:mprof-andamento}
Esta subseção apresenta o comprovante de co-orientações de Dissertação de Mestrado Profissional em andamento.\\

\bigskip
\begin{minipage}{\textwidth}
\includepdf[pages=-, scale=1,pagecommand={}]{\detokenize{GRUPO 1/Sub-Grupo 11/Comprovante Fake}}
\end{minipage}

\newpage
\subsection{Orientações de Trabalhos de Conclusão de Curso Concluídas}
\label{advisor:tcc}
Esta subseção apresenta o comprovante de orientações de Trabalhos de Conclusão de Curso concluídas.\\

\bigskip
\begin{minipage}{\textwidth}
\includepdf[pages=-, scale=1,pagecommand={}]{\detokenize{GRUPO 1/Sub-Grupo 11/Comprovante Fake}}
\end{minipage}

\newpage
\subsection{Orientações de Monitorias}
\label{advisor:monitoria}
Esta subseção apresenta o comprovante de orientações de Monitoria no período de 2014-2 a 2016-1 impressa diretamente da página do Sistema de Gerenciamento de Monitores Online (GMon)\footnote{URL: \url{https://www.cin.ufpe.br/~gmon}, Último acesso em XX/XX/XXXX}.\\
\bigskip
\begin{minipage}{\textwidth}
\includepdf[pages=-, scale=1,pagecommand={}]{\detokenize{GRUPO 1/Sub-Grupo 11/Comprovante Fake}}
\end{minipage}
\newpage
\subsection{Orientações de Trabalhos de Apoio Acadêmico}
\label{advisor:icc}
Esta subseção apresenta o comprovante de orientações de Trabalhos de Iniciação Científica.\\
\bigskip
\begin{minipage}{\textwidth}
\includepdf[pages=-, scale=1,pagecommand={}]{\detokenize{GRUPO 1/Sub-Grupo 11/Comprovante Fake}}
\end{minipage}

\newpage
\subsection{Orientações de Trabalhos de Apoio Acadêmico}
\label{advisor:bia}
Esta subseção apresenta o comprovante de orientações de Trabalhos de Apoio Acadêmico.\\
\bigskip
\begin{minipage}{\textwidth}
\includepdf[pages=-, scale=1,pagecommand={}]{\detokenize{GRUPO 1/Sub-Grupo 11/Comprovante Fake}}
\end{minipage}


%%%%%%%%%%%%%%%%%%%%%%%%%%%%%%%%%%%%%%%%%%%%%%%%%%%%%%%%%%%%%%%%%%%%%%%%%%%%%%%
% Subgrupo 1.2 - Participação em Comiss\~{o}es Examinadoras
%%%%%%%%%%%%%%%%%%%%%%%%%%%%%%%%%%%%%%%%%%%%%%%%%%%%%%%%%%%%%%%%%%%%%%%%%%%%%%%

\newpage
\subsection{Parecerista dos cursos de Sistemas de Informação da Avaliação de Cursos Superiores do Guia do Estudante (GE) 2015}
\label{app:2015-guia-estudante}
Esta subseção apresenta o comprovante da atuação como parecerista dos cursos de Sistemas de Informação da Avaliação de Cursos Superiores do Guia do Estudante (GE) 2015, promovido pela Editora Abril.\\
\bigskip
\begin{minipage}{\textwidth}
\includepdf[pages=-, scale=1,pagecommand={}]{\detokenize{GRUPO 1/Sub-Grupo 12/1/Comprovante Fake}}
\end{minipage}

%---

\newpage
\subsection{Comissão Avaliadora dos trabalhos na área de Ciências Exatas apresentados na 19ª Jornada de Iniciação Científica da FACEPE}
\label{app:2015-jic-facepe}
Esta subseção apresenta o comprovante da atuação como Membro da Comissão Avaliadora dos trabalhos na área de Ciências Exatas apresentados na 19ª Jornada de Iniciação Científica da Fundação de Amparo a Ciência e Tecnologia do Estado de Pernambuco (FACEPE).\\
\bigskip
\begin{minipage}{\textwidth}
\includepdf[pages=-, scale=1,pagecommand={}]{\detokenize{GRUPO 1/Sub-Grupo 12/2/Comprovante Fake}}
\end{minipage}

%---

\newpage
\subsection{Participação em Bancas de Trabalho de Conclusão de Curso}
\label{app:bancas-tcc}
Esta subseção apresenta o comprovante da atuação membro nas Bancas Examinadoras de Trabalho de Conclusão de Curso no Centro de Informática/UFPE.\\
\bigskip
\begin{minipage}{\textwidth}
\includepdf[pages=-, scale=1,pagecommand={}]{\detokenize{GRUPO 1/Sub-Grupo 12/4/Comprovante Fake}}
\end{minipage}
\newpage
\subsection{Participação em Banca de Dissertação }
\label{app:2015-msc-hslb}
Esta subseção apresenta o comprovante da atuação membro na Banca Examinadora de Defesa de Dissertação de Mestrado de. \\
\bigskip
\begin{minipage}{\textwidth}
\includepdf[pages=-, scale=1,pagecommand={}]{\detokenize{GRUPO 1/Sub-Grupo 12/5/Comprovante Fake}}
\end{minipage}

%---

\newpage
\subsection{Participação em Banca de Qualifica\c{c}\~ao / Proposta de Tese de Doutorado}
\label{app:2015-quali-phd-tmp}
Esta subseção apresenta o comprovante da atuação como membro na Banca Examinadora de Defesa de Proposta de Tese de Doutorado. \\
\bigskip
\begin{minipage}{\textwidth}
\includepdf[pages=-, scale=1,pagecommand={}]{\detokenize{GRUPO 1/Sub-Grupo 12/6/Comprovante Fake}}
\end{minipage}

%---

\newpage
\subsection{Participação em Banca de Tese de Doutorado }
\label{app:2015-phd-jcd}
Esta subseção apresenta o comprovante da atuação como membro na Banca Examinadora de Defesa de Tese de Doutorado. \\
\bigskip
\begin{minipage}{\textwidth}
\includepdf[pages=-, scale=1,pagecommand={}]{\detokenize{GRUPO 1/Sub-Grupo 12/7/Comprovante Fake}}
\end{minipage}

%---

\newpage
\subsection{Participação como Membro Externo da Comissão de Seleção para o Doutorado }
\label{app:2015-pgcomp-pdse}
Esta subseção apresenta o comprovante da atuação como membro Externo da Comissão de Seleção para o Doutorado. \\
\bigskip
\begin{minipage}{\textwidth}
\includepdf[pages=-, scale=1,pagecommand={}]{\detokenize{GRUPO 1/Sub-Grupo 12/8/Comprovante Fake}}
\end{minipage}
%%%%%%%%%%%%%%%%%%%%%%%%%%%%%%%%%%%%%%%%%%%%%%%%%%%%%%%%%%%%%%%%%%%%%%%%%%%%%%%
% Subgrupo 1.3 - Atividades de Ensino na Graduação e na Pós-Graduação
%%%%%%%%%%%%%%%%%%%%%%%%%%%%%%%%%%%%%%%%%%%%%%%%%%%%%%%%%%%%%%%%%%%%%%%%%%%%%%%



%%%%%%%%%%%%%%%%%%%%%%%%%%%%%%%%%%%%%%%%%%%%%%%%%%%%%%%%%%%%%%%%%%%%%%%%%%%%%%%
% Subgrupo 1.4 - Avaliação Didática Docente pelo Discente
%%%%%%%%%%%%%%%%%%%%%%%%%%%%%%%%%%%%%%%%%%%%%%%%%%%%%%%%%%%%%%%%%%%%%%%%%%%%%%%

\newpage
\subsection{Avaliação Didática Docente pelo Discente}
\label{evaluation:2015-1}
Esta subseção apresenta o comprovante da Avaliação Didática Docente pelo Discente para o período do primeiro semestre de 2015. \\
\bigskip
\begin{minipage}{\textwidth}
\includepdf[pages=-, scale=1,pagecommand={}]{\detokenize{GRUPO 1/Sub-Grupo 14/Comprovante Fake}}
\end{minipage}

%%%%%%%%%%%%%%%%%%%%%%%%%%%%%%%%%%%%%%%%%%%%%%%%%%%%%%%%%%%%%%%%%%%%%%%%%%%%%%%
% Subgrupo 2.1 - Produtividade de Pesquisa
%%%%%%%%%%%%%%%%%%%%%%%%%%%%%%%%%%%%%%%%%%%%%%%%%%%%%%%%%%%%%%%%%%%%%%%%%%%%%%%

\newpage
\subsection{Bolsista de produtividade em pesquisa e em inovação tecnológica}
\label{app:bolsista-prod}
Esta subseção apresenta o comprovante de Bolsista de produtividade em pesquisa e em inovação tecnolÓgica. \\
\bigskip
\begin{minipage}{\textwidth}
\includepdf[pages=-, scale=1,pagecommand={}]{\detokenize{GRUPO 2/Sub-Grupo 21/A/Comprovante Fake}}
\end{minipage}
\newpage
\subsection{Participação em Eventos Científicos (com apresentação de trabalho ou oferecimento de cursos, palestras ou debates}
\label{app:2015-cbsoft}
Esta subseção apresenta o comprovante da participação no Congresso Brasileiro de Software: Teoria e Prática (CBSoft'2015) com seus respectivos propósitos.
\includepdf[pages=-, scale=1,pagecommand={}]{\detokenize{GRUPO 2/Sub-Grupo 21/B/Comprovante Fake}}

\newpage
\subsection{Participação em Eventos Científicos (com apresentação de trabalho ou oferecimento de cursos, palestras ou debates}
\label{app:2015-enacomp}
Esta subseção apresenta o comprovante da participação no XII Encontro Anual de Computação (EnAComp) com seus respectivos propósitos.
\includepdf[pages=-, scale=1,pagecommand={}]{\detokenize{GRUPO 2/Sub-Grupo 21/B/Comprovante Fake}}
\includepdf[pages=-, scale=1,pagecommand={}]{\detokenize{GRUPO 2/Sub-Grupo 21/B/Comprovante Fake}}
\includepdf[pages=-, scale=1,pagecommand={}]{\detokenize{GRUPO 2/Sub-Grupo 21/B/Comprovante Fake}}

%---

\newpage
\subsection{Autoria de artigos completos publicados em anais de congresso, em jornais e revistas de circulação nacional e internacional na sua área}
\label{conf:2015-sbcup}
Esta subseção apresenta o comprovante da autoria de artigo completo publicado em anais do 7º Simpósio Brasileiro de Computação Ubíqua e Pervasiva (SBCUP)\footnote{\url{http://csbc2015.cin.ufpe.br/eventos_descricao/13}}, Evento do XXXV Congresso da Sociedade Brasileira de Computação.\\
\begin{minipage}
\includepdf[pages=-, scale=1,pagecommand={}]{\detokenize{GRUPO 2/Sub-Grupo 21/C/Comprovante Fake}}
\end{minipage}
%---

\newpage
\subsection{Arbitragem de Artigos Técnico-Científicos Nacionais e Internacionais na sua área de atuação}
\label{reviewer:2015-tsc}
Esta subseção apresenta o comprovante de arbitragem de artigos do periódico IEEE Transactions on Services Computing, editora IEEE Computer Society, ISSN: 1939-1374.
\includepdf[pages=-, scale=1,pagecommand={}]{\detokenize{GRUPO 2/Sub-Grupo 21/D/Comprovante Fake}}

%---

\newpage
\subsection{Coordenação e/ou Participação em Projetos Aprovados por Órgãos de Fomento}
\label{project:2015-facepe-pepe}
Esta subseção apresenta o comprovante de coordenação e/ou participação no Projeto \textit{``BIGStore - Evolução da plataforma Ustore para Armazenamento, Manipulação e Experimentação de Grandes Volumes de Dados''} aprovado pela Fundação de Amparo à Ciência e Tecnologia do Estado de Pernambuco (FACEPE), Edital FACEPE 23/2014 - PESQUISADOR NA EMPRESA DE PERNAMBUCO (PEPE).
\includepdf[pages=-, scale=1,pagecommand={}]{\detokenize{GRUPO 2/Sub-Grupo 21/E/Comprovante Fake}}

%---

\newpage
\subsection{Consultoria às Instituições de Fomento à Pesquisa, Ensino e Extensão}
\label{consulting:2015-facepe-pepe}
Esta subseção apresenta o comprovante de consultoria à Fundação de Amparo à Ciência e Tecnologia do Estado de Pernambuco (FACEPE) no Seminário de Avaliação Final dos Projetos Aprovados no Edital 10.2/2012 - PAPPE INTEGRAÇÃO 04a RODADA.
\includepdf[pages=-, scale=1,pagecommand={}]{\detokenize{GRUPO 2/Sub-Grupo 21/F/Comprovante Fake}}

%---

\newpage
\subsection{Prêmios Recebidos pela Produção Científica e Técnica}
\label{award:2014}
Esta subseção apresenta o comprovante de consultoria à Fundação de Amparo à Ciência e Tecnologia do Estado de Pernambuco (FACEPE) no Seminário de Avaliação Final dos Projetos Aprovados no Edital 10.2/2012 - PAPPE INTEGRAÇÃO 04a RODADA.
\includepdf[pages=-, scale=1,pagecommand={}]{\detokenize{GRUPO 2/Sub-Grupo 21/G/Comprovante Fake}}

%%%%%%%%%%%%%%%%%%%%%%%%%%%%%%%%%%%%%%%%%%%%%%%%%%%%%%%%%%%%%%%%%%%%%%%%%%%%%%%
% Subgrupo 2.2 - Produção Científica
%%%%%%%%%%%%%%%%%%%%%%%%%%%%%%%%%%%%%%%%%%%%%%%%%%%%%%%%%%%%%%%%%%%%%%%%%%%%%%%

\newpage
\subsection{Trabalhos Publicados em PeriÓdicos Especializados do País ou do Exterior}
\label{journal:2015-medinfo}
Esta subseção apresenta o comprovante de publicação do artigo ```(Br-SCMM) Brazilian Smart City Maturity Model: A Perspective from the Health Domain'' no Studies in Health Technology and Informatics, v. 216 (MEDINFO 2015: eHealth-enabled Health), p. 983, 2015, doi: 10.3233/978-1-61499-564-7-983.
\includepdf[pages=-, scale=1,pagecommand={}]{\detokenize{GRUPO 2/Sub-Grupo 22/Comprovante Fake}}

%%%%%%%%%%%%%%%%%%%%%%%%%%%%%%%%%%%%%%%%%%%%%%%%%%%%%%%%%%%%%%%%%%%%%%%%%%%%%%%
% Grupo 3: Atividades de Extensão
%%%%%%%%%%%%%%%%%%%%%%%%%%%%%%%%%%%%%%%%%%%%%%%%%%%%%%%%%%%%%%%%%%%%%%%%%%%%%%%

%%%%%%%%%%%%%%%%%%%%%%%%%%%%%%%%%%%%%%%%%%%%%%%%%%%%%%%%%%%%%%%%%%%%%%%%%%%%%%%
% Subgrupo 3.1 - Coordenação e Orientação
%%%%%%%%%%%%%%%%%%%%%%%%%%%%%%%%%%%%%%%%%%%%%%%%%%%%%%%%%%%%%%%%%%%%%%%%%%%%%%%

\newpage
\subsection{Coordenação e Orientação de Atividades de Extensão}
\label{extensao:2015}
Esta subseção apresenta o comprovante de Coordenação e Orientação de Atividades de Extensão.
\includepdf[pages=-, scale=1,pagecommand={}]{\detokenize{GRUPO 3/Sub-Grupo 31/Comprovante Fake}}

%%%%%%%%%%%%%%%%%%%%%%%%%%%%%%%%%%%%%%%%%%%%%%%%%%%%%%%%%%%%%%%%%%%%%%%%%%%%%%%
% Subgrupo 3.2 - Coordenação de Eventos e Conferencista
%%%%%%%%%%%%%%%%%%%%%%%%%%%%%%%%%%%%%%%%%%%%%%%%%%%%%%%%%%%%%%%%%%%%%%%%%%%%%%%

\newpage
\subsection{Comissão Organizadora de Eventos Internacional, Nacional, Regional ou Local}
\label{reviewer:2015-sbcars}
Esta subseção apresenta o comprovante de Comissão Organizadora do 9th Brazilian Symposium on Software Components, Architectures and Reuse (SBCARS 2015).
\includepdf[pages=-, scale=1,pagecommand={}]{\detokenize{GRUPO 3/Sub-Grupo 32/Comprovante Fake}}

%%%%%%%%%%%%%%%%%%%%%%%%%%%%%%%%%%%%%%%%%%%%%%%%%%%%%%%%%%%%%%%%%%%%%%%%%%%%%%%
% Grupo 4: Atividades de Formação e Capacitação Acadêmica
%%%%%%%%%%%%%%%%%%%%%%%%%%%%%%%%%%%%%%%%%%%%%%%%%%%%%%%%%%%%%%%%%%%%%%%%%%%%%%%

\newpage
\subsection{Atualização e Cursos de Capacitação ou Extensão na área de Conhecimento ou Afins com no Mínimo 40h}
\label{courses:2015}
Esta subseção apresenta o comprovante de Atualização e Cursos de Capacitação ou Extensão no curso Machine Learning (Coursera).
\includepdf[pages=-, scale=1,pagecommand={}]{\detokenize{GRUPO 4/Comprovante Fake}}

%%%%%%%%%%%%%%%%%%%%%%%%%%%%%%%%%%%%%%%%%%%%%%%%%%%%%%%%%%%%%%%%%%%%%%%%%%%%%%%
% Grupo 5: Atividades Administrativas
%%%%%%%%%%%%%%%%%%%%%%%%%%%%%%%%%%%%%%%%%%%%%%%%%%%%%%%%%%%%%%%%%%%%%%%%%%%%%%%

\newpage
\subsection{Atividades Administrativas, Membro de Comissão Temporária}
\label{committee:temp-2015}
Esta subseção apresenta o comprovante de Atividades Administrativas, Membro de Comissão Temporária.
\includepdf[pages=-, scale=1,pagecommand={}]{\detokenize{GRUPO 5/Comprovante Fake}}

%------------------------------------------------------------------------------

\newpage
\subsection{Atividades Administrativas, Coordenador de Curso Pós-Graduação \textbf{strictu sensu}}
\label{committee:postgrad-2015}
Esta subseção apresenta o comprovante de Atividades Administrativas, Coordenador de Curso Pós-Graduação \textbf{strictu sensu}.
\includepdf[pages=-, scale=1,pagecommand={}]{\detokenize{GRUPO 5/Comprovante Fake}}

%------------------------------------------------------------------------------

\newpage
\subsection{Atividades Administrativas, Membro de Núcleo Docente Estruturante}
\label{committee:nde-2015}
Esta subseção apresenta o comprovante de Atividades Administrativas, Membro de Núcleo Docente Estruturante.
\includepdf[pages=-, scale=1,pagecommand={}]{\detokenize{GRUPO 5/Comprovante Fake}}

\newpage
\subsection{Atividades Administrativas, Membro de Núcleo Docente Estruturante}
\label{committee:nde-2014}
Esta subseção apresenta o comprovante de Atividades Administrativas, Membro de Núcleo Docente Estruturante.
\includepdf[pages=-, scale=1,pagecommand={}]{\detokenize{GRUPO 5/Comprovante Fake}}

%------------------------------------------------------------------------------

\newpage
\subsection{Atividades Administrativas, Membro de Colegiados de Curso de Graduação e PÓs-Graduação}
\label{committee:colegiado-postgrad-2015}
Esta subseção apresenta o comprovante de Atividades Administrativas, Membro do Colegiado da PÓs-Graduação (em curso).
\includepdf[pages=-, scale=1,pagecommand={}]{\detokenize{GRUPO 5/Comprovante Fake}}

\newpage
\subsection{Atividades Administrativas, Membro de Colegiados de Curso de Graduação e PÓs-Graduação}
\label{committee:colegiado-cc-2015}
Esta subseção apresenta o comprovante de Atividades Administrativas, Membro do Colegiado do Curso de Graduação em Ciência da Computação.
\includepdf[pages=-, scale=1,pagecommand={}]{\detokenize{GRUPO 5/Comprovante Fake}}

\newpage
\subsection{Atividades Administrativas, Membro de Colegiados de Curso de Graduação e PÓs-Graduação}
\label{committee:colegiado-si-2015}
Esta subseção apresenta o comprovante de Atividades Administrativas, Membro do Colegiado do Curso de Graduação em Sistemas de Informação em 2015.
\includepdf[pages=-, scale=1,pagecommand={}]{\detokenize{GRUPO 5/Comprovante Fake}}

\newpage
\subsection{Atividades Administrativas, Membro de Colegiados de Curso de Graduação e PÓs-Graduação}
\label{committee:colegiado-si-2014}
Esta subseção apresenta o comprovante de Atividades Administrativas, Membro do Colegiado do Curso de Graduação em Sistemas de Informação em 2014.
\includepdf[pages=-, scale=1,pagecommand={}]{\detokenize{GRUPO 5/Comprovante Fake}}

%%%%%%%%%%%%%%%%%%%%%%%%%%%%%%%%%%%%%%%%%%%%%%%%%%%%%%%%%%%%%%%%%%%%%%%%%%%%%%%
% \newpage
% \subsection{Portaria de Progressão}
% \label{app:2014-portaria-progressao}
% \includepdf[pages=-, scale=1,pagecommand={}]{\detokenize{GRUPO 1/20141205_Portaria-de-Progressao-Funcional_5929-2014.pdf}}


\end{document}

%%% EOF
